\chapter{Background}
\label{chap:background}

In this chapter we are going to explain the most important concepts of the 
upcoming 5G standard, talking about the Virtual Network Functions (VNFs) and 
how first experiments with this new technology already begun. Finally we are 
going to briefly introduce and explain the technologies involved and their role.

\section{The upcoming connectivity standard: 5G}
The continuous innovation in the mobile network connectivity is leading to the
creation of a new standard, the 5G, which is estimated to arrive to the market
consumer in 2020\todo{CITE}. Lead by the Next Generation Mobile Network (NGMN)
alliance, a group composed by the major players in the field of mobile
connectivity, the 5G aims to offer not only at the end-user a new way to browse 
the web, download and watch interactive content but also to create an ad-hoc 
solution for Machine-to-Machine (M2M) data traffic, which is increasing more 
than ever thanks to the spreading of IoT devices, which sensors need to 
continuously send data to servers/data-centers. Relatively to LTE\todo{Explain 
acronym}, 5G points to have data rates $10$ times better, with $10$ times 
smaller end-to-end latency and an increased connection density by $100$ 
times\todo{CITE}.

\begin{table}[t]
\centering
\resizebox{\textwidth}{!}{%
  \begin{tabular}{p{4,5cm}|p{5,5cm}|p{5cm}}
\textbf{Attribute}                                                    & 
\textbf{LTE capability}                                                          
                                                                           & 
\textbf{Improvement needed to meet NGMN requirements}                 \\ \hline
\textbf{Data rate (per user)}                                         & Up to 
100 Mb/s on average Peaks of 600 Mb/s (Cat 11/12).                               
                                                                     & 10X 
expected on average and peak rates and 100X expected on cell edge \\
\textbf{End-to-end latency}                                           & 10 ms 
for two-way RAN (pre-scheduled). Typically, up to 50 ms end-to-end if other 
factors are considered (e.g., transmission, CN, internet, proxy servers). & 10X 
(smaller)                                                         \\
\textbf{Mobility}                                                     & 
Functional up to 350 km/h (for certain bands up to 500 km/h). No support for 
civil aviation.                                                                & 
1.5X                                                                  \\
\textbf{\begin{tabular}[c]{@{}l@{}}Connection\\ density\end{tabular}} & 
Typically $\sim$2,000 active users/km2.                                          
                                                                           & 
100X                                                                 
\end{tabular}%
}
\caption[5G improvement over LTE]{An extract from the official 5G white paper 
illustrating the improvements of 5G relatively to LTE connections.}
\label{chap:intro:table:ltevs5g}
\end{table}

Part of the new requirements can be satisfied using a large radio spectrum with
higher frequencies. The utilization of higher frequencies, though, mean that the
radio signals can be easily disrupted by physical objects, like buildings and
many geographical elements (such as hills and mountains), clashing with the
expectation of an ever-reachable connectivity. It is here where virtualization
plays an important role. In fact, the re-design of some network components today
existing via hardware can transform a monolithic networking approach to a 
modular one, exploiting the flexibility that Virtual Network Functions 
(VNFs)\todo{CITE VNF-P: A Model for Efficient Placement of Virtualized Network 
Functions} can offer thanks to virtualization, closing the gap to the use-case 
fulfillment defined by the NGMN alliance\todo{CITE: NGMN View on 5G 
architecture} which require a decreased time to set up and deploy network 
services (specifically, from 90 hours to 90 minutes\todo{Search this 
requirement, check 90h 'cause I'm not sure about it}).

\subsection{5G architecture}

With the constraints placed on the requirements formulated by the NGMN alliance,
5G envisage a multi-layered architecture, based on three main layers:\todo{CITE:
  NGMN View on 5G architecture}
\begin{itemize}
\item \textbf{infrastructure resource layer}: physical resources that are 
exposed via a virtualized interface, and that can be monitored using specific 
APIs
\item \textbf{business enablement layer}: where a library of functions and
  deployment is contained, and its configuration is accessible via APIs
\item \textbf{business application layer}: layer that contains specific
  applications and services of the operator
\end{itemize}

\todo{Add three-layer image here?}

This separation in layers allows to easily manage multiple identities 
differently: an Internet Service Provider (ISP) could manage a multitude of 
physical infrastructures and have multiple business enablements dislocated 
along an entire continent for example, but it could decide to have only a 
single centralized business application deployment that manage all the other 
layers resources.

\subsubsection{Network slicing}
The role of a ``network slice'' in a 5G architecture is to specifically handle 
the Control-plane\todo{Explain what a control plane is?} of a particular 
service (e.g. smartphones traffic, autonomous driving, massive IoT), deploying 
resources in a manner that assure the required latency, security and 
reliability. While some very peculiar legacy services could require specific 
hardware, the common resources between services could be shared in a 
virtualized way, providing auto-scaling capabilities in services that are under 
heavy network pressure.

\section{The VIBES project}
 
 This thesis is part of the VIBES project\todo{Talk about VIBES project, add
   some reference, explain what it is.}, where the necessity for better TCP/IP
 transmission through satellite connections is the main requirement. To reach
 this goal, the project specifications suggest to exploit the 5G incoming
 technology and use the NFV-MANO architecture to perform first packets
 elaboration and performance improvement and finally TCP/IP satellite chunk
 optimization with the Performance Enhancing Proxy (PEP). The VIBES project
 proposed five technical requirements:
\begin{enumerate}
 \item Analysis of the applicability of current and new Internet protocols in
   the proposed VNF-PEP architecture
 \item Implementation of a VNF-PEP prototype
 \item Building of a PoC test platform
 \item VNF-PEP validation and performance tests on 5G use-cases
 \item Demonstration test-bed management
\end{enumerate}

\begin{figure}[t]
 \centering
 \includegraphics[scale=1]{vibes_logo}
 \caption{ViBeS project logo}
 \label{chap:background:img:vibes_logo}
\end{figure}


\subsection{VNF-PEP architecture and internet protocols}

The analysis of this topic revealed to be trivial: since internet has many
different protocols that would become infeasible to support all of them at the
same time, packet encapsulation present itself as the only feasible solution:
every packet incoming in the VNFs has already been encapsulated by a generic
packet encapsulator/decapsulator,
\todo{Also I think that the main purpose of UDP encapsulation is not to hide 
the protocol used on the edges but 1) use a ``quick protocol to exchange data 
among VNFs and 2) we are hiding the path not the protocol itself. As we were 
discussing, we are creating some sort of proxy, so the aim will be the same 
even if we support a plethora of protocols.''} making the whole architecture 
independent from the protocol a particular flux of data. To achieve this, 
several solutions have been studied, and at the end packet encapsulation with 
TCP split (for TCP sessions) have been chosen. The rationale that guided us on 
this choice is described in Chapter~\ref{chap:vnf_ns_impl}. \todo{Update 
reference with the section of the explanation}

\subsection{VNF-PEP prototype}

Since the requirement for the whole system (MANO+PEP) were too challenging, the 
goal shifted into creating a MANO test-bed and a working NFVI, excluding PEP. 
The VNF architecture was shaped following the container orchestrator we decided 
to use. An more detailed architectural implementation can be found at 
Chapter~\ref{chap:archimpl}.

% \section{Technologies used}

\vspace{0.5cm}

Starting with the first step of creating a MANO able to process incoming data
packets through VNF functions, we encountered that many networking tools already
present in the market required some tweaking and some integration, shifting our
goal to create a complete European Telecommunications Standards Institutes
(ETSI) Management and orchestrator (MANO) test-bed instead, following the
specifications suggested in the RFC 7665, thus implementing only the first three
requisites, without digging in the satellite data flow optimization. In
particular, we discovered how, these tools, were suitable to create ETSI MANO
and VNFs using virtual machine or exploiting cloud technologies, while they were
not designed with enough flexibility an integration with Docker. In the next
chapter we are going to take a deep analysis of the cited technologies 
(Section~\ref{chap:prjan:sec:tech}, in order to provide to the reader enough 
context to be able to understand our design choices we are going to describe 
later.

\noindent With this in mind, we performed a requirements analysis described in
the next chapter.\todo{This should be put at the end of the introduction.}
 
