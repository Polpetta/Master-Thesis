\chapter{Architectural implementation}
\label{chap:archimpl}
 
In this chapter we will dig into the actual architectural implementation, 
seeing all the problems related to it and how they were solved.

\section{Adopting Openstack for container orchestration}

Our first step was to try to use Openstack to create a container orchestration. 
The reason behind this approach was simple: Openbaton, on paper, comes with an 
out-of-the-box support for Openstack. In fact, it natively supports Openstack 
PoPs, giving to us a great possibility to reuse the two frameworks and to ease 
the development phase of our software.
\subsection{Openstack for developers: Devstack}
Openstack is composed by modules, giving the possibility to the user who wants 
to install it to choose only the components it really needs. So the first thing 
we decided was to perform a selection of these modules between the 33 available 
in the installation page. The installation of this components span between 
multiple machines (nodes), therefore creating a whole cluster of resources. 
Since we considered too time-consuming to install Openstack in a distribuited 
manner, we considered to test first the developer version, called 
\emph{Devstack}. Devstack offers the possibility to install a subset of modules 
in the local machine, and to leverage a Openstack environment able to perform 
the same operations of the production-grade installation. Thus, we starting 
configurating the Devstack installation, installing first of all the Nova 
module, that is required to perform computer operations such as the launch of 
virtual machines, and lately we installed Taker too.\todo{exaplain Taker 
functionalities}

\paragraph*{Devstack configuration}
The Openstack installation is not an easy one, particularly for newcomers, that 
have to deal with a great set of tools and not-always-clear instructions. Here 
we will describe how we managed to get a single-node deployment up and running.

First of all, Devstack needs a machine with at least 16GB of ram and 50GB of HD 
space.
We decided to install these components:
\begin{itemize}
 \item Keystone (for identity management)
 \item Object storage
 \item Compute
 \item Tacker (we tested this module to see is Openstack was a suitable VIM too)
\end{itemize}

Using Devstack is a simple operation, but the time required to reach a correct 
configuration is very high. In particular, it required from 20 minutes to 1 
hour to install Devstack we a particular configuration, without knowing if the 
final installation was working or not. This trial-and-error approach consumed a 
considerable amount of time, but it was the only possibility we had.
