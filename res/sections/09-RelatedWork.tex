\chapter{Related Work}
\label{chap:rel_wk}


The possibility to solve the increasing network capacity necessities and to make
the network infrastructure lighter and faster to deploy has rouse the interests
of many internet service providers (ISP).

\section{NFV alternative implementations}

In literacy, several NFV implementations and Network Slicing solutions have been
proposed and some of them will be discussed in the following chapter. For
example, an opportunity identified in~\cite{nogales2018nfv} shows that ETSI MANO
implementations can be deployed inside Unmanned Aerial Vehicles (UAVs), giving
the possibility to rapidly deploy network connectivity in areas where the common
communication systems are temporarily unavailable. This can be very useful in
places where natural disasters occurred and there is the necessity to restore
the communications within a very short span of time. On top of that, this system
is flexible enough to be able to provide different kind of services, in addition
to simple broadband connectivity. The only downside of this approach are the
constrained system sources: the authors had to install the VIM (in particular,
Openstack) in a separated machine, located on the ground. A solution based only
on containerization, instead, could lead to interesting results, since
Kubernetes does not require too much resources \textit{per se} and in literature
is possible to find solutions of Kubernetes installations performed on machines
with the same hardware installed on the UAVs~\cite{pahl2016container}.

\section{EPC virtual implementations}

A virtualized EPC implementation, designed to allow mobile broadband services in
the UMTS/LTE world, has been described in~\cite{hawilo2014nfv}, showing how this
implementation is able, thanks to entities grouping, to perform better than a
traditional EPC installation. In~\cite{afolabi2017end} is possible to read
another vEPC implementation, where the Wireless Network Virtualization (WNV),
SDN and NFV (identified from the authors as key elements in order to implement
the 5G architecture) are virtualized, and ported as a service (defined as
EPCaaS) across multi-domains clouds. The realization of such framework is
pursued thanks to end-to-end network slicing.

EPC and ETSI MANO definitions are not the only solutions to have been explored
at this time. An SDN/NFV-based end to end network slicing for 5G has been
created in~\cite{chartsias2017sdn}, where the authors provided examples to
demonstrate how end-to-end services can be provisioned using SDN networking, in
particular with controllers like such as OpenDaylight, Pox, Ryu ONOD and
Floodlight. The last one was studied by us too during the technologies analysis,
and a description is available in
Section~\ref{chap:prjan:sec:tech:sub:other:sub:floodlight}.
%TODO section missing...write it down!!

\section{Network slicing security concerns}

Another important aspect related to 5G networking and network slice is the one
regarding security. Security in this context plays a fundamental role, since
secure communications depends on the security of the overall system.
In~\cite{kotulski2017end} the authors carry out a deep analysis about the
network slice security, putting an emphasis on how this new element (the network
slice) brings additional security questions and problems, especially about
isolation. Moreover, slicing in 5G introduces the concept of shared services,
that is contrary with the definition of isolation itself. In particular, 
resource isolation gives the possibility to avoid attacks such us slice 
resources exhaustion. Additionally, the authors give a critic regarding the 
ETSI NFV-MANO standard making too general assumptions regarding the whole 
system. Finally, a Service-Level Agreement (SLA) based orchestration criteria 
between network slices is proposed as a viable solution to solve the security 
problems about isolation.

\vspace{1cm}

\noindent It is interesting to note that, between all the possible and 
alternative related works we have been able to find, none of these used an only
container-based approach.
