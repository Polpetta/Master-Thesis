\chapter{Related Work}
\label{chap:rel_wk}


The possibility to solve the increasing network capacity necessities and to make
the network infrastructure lighter and faster to deploy has rouse the interests
of many internet service providers (ISP).

\section{NFV alternative implementations}

In literacy, several NFV implementations and Network Slicing solutions have been
proposed and some of them will be discussed in the following chapter. For
example, an opportunity identified in~\cite{nogales2018nfv} shows that ETSI MANO
implementations can be deployed inside Unmanned Aerial Vehicles (UAVs), giving
the possibility to rapidly deploy network connectivity in areas where the common
communication systems are temporarily unavailable. This can be very useful in
places where natural disasters occurred and there is the necessity to restore
the communications within a very short span of time. On top of that, this system
is flexible enough to be able to provide different kind of services, in addition
to simple broadband connectivity. The only downside of this approach are the
constrained system sources: the authors had to install the VIM (in particular,
Openstack) in a separated machine, located on the ground. A solution based only
on containerization, instead, could lead to interesting results, since
Kubernetes does not require too much resources \textit{per se} and in literature
is possible to find solutions of Kubernetes installations performed on machines
with the same hardware installed on the UAVs~\cite{pahl2016container}.

\section{EPC virtual implementations}

A virtualized EPC implementation, designed to allow mobile broadband services in
the UMTS/LTE world, has been described in~\cite{hawilo2014nfv}, showing how this
implementation is able, thanks to entities grouping, to perform better than a
traditional EPC installation. In~\cite{afolabi2017end} is possible to read
another vEPC implementation, where the Wireless Network Virtualization (WNV),
SDN and NFV (identified from the authors as key elements in order to implement
the 5G architecture) are virtualized, and ported as a service (defined as
EPCaaS) across multi-domains clouds. The realization of such framework is
pursued thanks to end-to-end network slicing.

EPC and ETSI MANO definitions are not the only solutions to have been explored
at this time. An SDN/NFV-based end-to-end network slicing for 5G has been
created in~\cite{chartsias2017sdn}, where the authors provided examples to
demonstrate how end-to-end services can be provisioned using SDN networking, in
particular with controllers like such as OpenDaylight, Pox, Ryu ONOD and
Floodlight. The last one was studied by us too during the technologies analysis,
and a description is available in
Section~\ref{chap:prjan:sec:tech:sub:other:sub:floodlight}.
%TODO section missing...write it down!!

Compared to our solution, these implementations does not present an unified
framework that gives the possbility orchestrate different network slices
consistently. Adopting Kubernetes gives the advantage to use the Kubernetes
Federation feature~\cite{kubeFederation}, enabling the possibility to simplify
the management of multiple network slice in end-to-end services.

\section{Network slicing security concerns}

Another important aspect related to 5G networking and network slice is the one
regarding security. Security in this context plays a fundamental role, since
secure communications depends on the security of the overall system.
In~\cite{kotulski2017end} the authors carry out a deep analysis about the
network slice security, putting an emphasis on how this new element (the network
slice) brings additional security questions and problems, especially about
isolation. Moreover, slicing in 5G introduces the concept of shared services,
that is contrary with the definition of isolation itself. In particular,
resource isolation gives the possibility to avoid attacks such us slice
resources exhaustion. Additionally, the authors give a critic regarding the ETSI
NFV-MANO standard making too general assumptions regarding the whole system.
Finally, a Service-Level Agreement (SLA) based orchestration criteria between
network slices is proposed as a viable solution to solve the security problems
about isolation. We tried to mitigate the security consern presented above using
the integrated Role Base Access Control system (RBAC) and other security policy
bundled in by default in Kubernetes.

\section{Provision capabilities}
We noted how, in some other NFV/ETSI-MANO proposals like~\cite{soares2015toward}
and~\cite{abujoda2015midas}, the implementations relays on a static provision
system, where the SFC paths are somehow defined at deploy time, and can't change
without decomissioning, editing and deploying the chain again. Our proposal,
instead, is able to support chain editing on-the-fly, allowing NFV to dynamically
change SFP.

\vspace{1cm}

\noindent It is interesting to note that, between all the possible and 
alternative related works we have been able to find, none of these used an only
container-based approach.

\subsection{Other SFC implementation}
Since the SFC technology was proposed many work came out proposing different implementation solutions. In this part there is an overview of some of them.

In~\cite{GhaznaviSAB16} proposed optimization model to calculate the optimal
location of VNFs based on different metrics such as bandwidth, load balancing
and routes. This decomposes SFC on performance requirements eliminating the
throughput bound of a chain to a single VNF or a single physical machine.
Mathematical models developed was tested in a simulation enviroment. A critical
enhancement to this work could be to test algorithm in a real environment, under
different conditions of traffic load: if it efficiant even in this senario it
can be useful to the orchestration and management part of the ecosystem because
it can reduce resource waste. 

Other implementation tested in a simulated environment
are~\cite{csoma2014escape} and~\cite{kim2016evaluations}. In both of them the
network functions are deployed on Mininet. The matter work provide a full
implementation of the SFC environment, including SFC configuration, allocating
VNFs into the physical resources, flow routing across the VNFs based on
policies, and provisioning live management information on operating VNF
instances, but no performance evaluation of the work is presented. In the
latter case chains are identified by VLAN tag and the VNF chaining is created
exploiting only using Floodlight controllers. 

Another noteworthy work is~\cite{xia2015optical} that discuss the packet
delivery through function chaining in high capacity network exploiting aggregate
flow steering. The architecture proposed make use of SDN and cloud platform to
deploy VMs on which run VNFs assuming that the SDN and network element are
connected using optical circuit switching, taking advantage of Wave Selective
Switchig (WSS), to route (switch) signals between optical fibres on a
per-wavelength basis. Thank to the implementation of the SDN controller and the
communication between the previous element and the VNFM, this architecture is
able to provide good capabilities in terms of flexibilities. Another element
that helps the flexibility is the high performance link: in this way they get
rid of problems in terms of connection time and bandwidth. However, in the paper
authors assumes really specific hardware in that underlying infrastructure that
connects the overall system: this is not suitable to be deployed in a general
purpose infrastructure or using IaaS. It will be interesting to evaluated the
same work on different instrumentation. A clashing aspect is the time to make a
VM up and running compared to times of servvice chaining configuration/packet
delivery. Using containers the instantiation of virtual functions should employ
less time.

A rilevant work on function chaining in network is~\cite{trajkovska2017sdn}. In
this work author efforts are concentrated on joining SDN and VNF technologies to
create SFCs. They developed an SDN-driven SDK, a virtual traffic classifier to
analyze traffic in real-time and decide the optimal chain for it and a VNF
specialized on video transdecoding, used event to evaluate the performances of
the whole system. Chaining is performed exploiting layer 2 connectivity,
using OpenvSwitch as virtual network svitches an OpenDayLight in order to
route the traffic on a set of policies. However the solution does not provide
any mechanism to automatic deployment of chains and services: this can be an
issue in particular in case of VNF fault. Orchestration and some mechanisms of
fault tollerance are critical in particular in a real environment, enabling
autoscaling on user demand or traffic load. Automated deployment also make
possible the recreation of a virtual machine in case of fault. As a matter of
fact that a virtual function can fail depending on an erroneus load balancing
policy or too heavy computational load and it is critical in order to do not
interrupt end-to-end communication to bring up the VNF as soon as possible. This
recreation process can be slow due to the usage of VMs to deploy the
functions, that could take minutes to startup.

The paper~\cite{kriti2017dnfc} introduce the possibility using container in SFC
deployment. The chain of container is created exploiting Openflow to create flow
to route traffic to the containers, that have to be installed on the host
machine. This work however does present any form of management or orchestration
of the chain, neither an automatic deployment. Moreover it requires installation
of part of the system on the host machine: in this way there is not a fully
virtualized environment. The matter issue has some implication: the VNF
has a strong relation on the host on which is running and the system could not
be simply be moved from one machine to another as it could be done only using
containers. Finally this paper does not provide any form of system evaluation
neither in terms of performances nor in terms of resilience. Fault recovering
and autoscaling could be achieved only improving the system introducing other
technologies, such as Docker Compose, Kubernetes or Docker Swarm. 






