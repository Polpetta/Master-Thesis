\chapter{Related Work}
\label{chap:rel_wk}

The possibility to solve the constant network capacity requests and to make
network infrastracture lighter and faster to deploy has rouse the interests of
many internet service providers (ISP). In literacy, several NFV implementations
and Network Slicing solutions have been proposed and some of them will be
discussed in the following chapter. An opportunity identified
in~\cite{nogales2018nfv} shows that ETSI MANO implementations can be deployed
inside Unmanned Aerial Vehicles (UAVs), giving the possibility to rapidly deploy
network connectivity in areas where the common communication systems are
temporarily unavailable. This can be very useful in places where natural
disasters occurred and there is the necessity to restore the communications
within a very short span of time. On top of that, this system is flexible enough
to be able to provide different kind of services, in addition to simple
broadband connectivity. The only downside of this approach are the constrained
system rources: the authors had to install the VIM (in particular, Openstack) in
a separated machine, located on the ground. A solution based only on
containerization, instead, could lead to interesting results, since Kubernetes
does not require too much resources \textit{per se} and in literature is
possible to find solutions of Kubernetes installations performed on machines
with the same harware installed on the UAVs~\cite{pahl2016container}.

A virtualized EPC implementation has been describe in~\cite{hawilo2014nfv}
% TODO talk about vEPC



It is interesting to note that, between all the possibile and alternative
related works we have been albe to find, none of these used Docker as an
alternative virtualization approach.
