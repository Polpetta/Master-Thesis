\chapter{Conclusions}
\label{chap:conclusions}

In this thesis we have seen an ETSI MANO and RFC 7665 implementation proposal. 
We described the technological components needed to build it, and we presented 
the requirements of the whole implementation. Then we described the challenges 
we encountered and how we solved it, taking choices valuating their pros and 
cons.

Finally, we corroborated our implementation with tests and metrics, showing 
that a containerized approach is faster and more reliable and faster than a 
virtualized one, and illustrating that latency in communications through SFCs, 
while there is room for improvement, are decent considered that this is just 
a test-bed implementation.

In conclusion, we can say that the ETSI MANO specification, coupled with the 
RFC 7665 proposal and the new virtualization technologies available in the 
market could change the shape of future mobile communications, leading to the 
creation of a 5G able to either satisfy the challenges of the present, such as 
IoT communications, media streaming, and real-time data, with the challenges of 
the future, as smart dust and flying autonomous vehicles emerge as the next 
possible innovations where connectivity will play an essential role.

\section*{Personal conclusions}

Generally speaking, I have never designed and though a software architecture of
these dimensions. The multiple attempts and issues I faced during the thesis
taught me more that I would have though, showing that, in life, you never end to
learn something new. Its for this reason that I think this project is something
more than a simple thesis, because I was put in front of challenges that seemed
over my possibilities at first, but that revealed to be no more than any other
common problem after having spent time analyzing, studyng and understanding
them. At the end, I consider every second of my time spent on this thesis worth
it, and it gave me the confidence that every limit you face in life can be
broken down.
