\chapter{Conclusions}
\label{chap:conclusions}

In this thesis we have seen an ETSI MANO and RFC 7665 implementation proposal. 
We described the technological components needed to build it, and we presented 
the requirements of the whole implementation. Then we described the challenges 
we encountered and how we solved it, taking choises valuating their pros and 
cons.

Finally, we corroborated our implementation with tests and metrics, showing 
that a containerized approach is faster and more reliable and faster than a 
virtualized one, and illustrating that latency in communications through SFCs, 
while there is room for improvement, are decent considered that this is just 
a test-bed implementation.

In conclusion, we can say that the ETSI MANO specification, coupled with the 
RFC 7665 proposal and the new virtualization technologies avaiable in the 
market could change the shape of future mobile communications, leading to the 
creation of a 5G able to either saddisfy the challenges of the present, such as 
IoT communications, media streaming, and real-time data, with the challenges of 
the future, as smart dust and flying autonomous vehicles emerge as the next 
possible innovations where connectivity will play an essential role.
