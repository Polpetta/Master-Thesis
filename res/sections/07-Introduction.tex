 % This is just a draft, nothing suitable for the final product: sources are in
 % form of a todo
 
 % --- INTRODUCTION ---
\chapter{Introduction}
\label{chap:intro}

 Through the history of internet, many factors changed and new innovations came,
 but at the end of the day a fundamental key remained: connection performances
 are, indeed, what the users really want.\todo{Idea: talk about how TCP/IP
   performs quite well in wired connections, while in wireless one, such as
   wireless/satellite, sucks ass}
With the arrival of mobile connectivity, wireless networks started to have an 
important role in everyone's life. In this context network efficiency plays 
an important role with the development of new connectivity standards (e.g 2G, 
3G, LTE) and the creation of common, ad-hoc versions, transmission protocols 
(i.e. TCP versions specialized for exploit the characteristic of 
wireless networks). It is for this reason that, beside improved radio 
communications, back-end infrastructure began to be refined too. During their 
communication to the destination packets are elaborated, compressed and 
processed in order to make their transmission as fast as possible.
On top of that, mobile traffic is continuously surging every year, with the 
backbones having more and more data to deliver, making in-hardware backbone 
solutions less efficient. Their time-to-deploy, in fact, is very high compared 
to the requirements of the network. With this trend, in-hardware solutions 
will not be able to keep up. Is with this considerations that a software 
approach is being proposed to become the new way to deploy network 
components. While less efficient, they are faster to bring them up, offering an 
unprecedented flexibility\todo{CITE: SDN/NFV-Based Mobile Packet Core Network 
Architectures: A Survey}.
 Furthermore, new virtualization technologies such as Cloud computing and Docker
 are becoming more and more mature for being employed in large-scale production
 environments, where new frameworks, called orchestrators (e.g. Kubernetes,
 Docker Swarm), allow to easily manage the application lifecycle, i.e.
 deployment, scale up/down and finally, removal.
 
 \section{Document organization}
 
 This thesis is organized as follows:
 \begin{itemize}
  \item Chapter~\ref{chap:background} describes background information about 5G 
and the technologies used to build an ETSI compliant architecture
  \item Chapter~\ref{chap:prjan} explains the analysis performed on the 
requirements provided and the design choices performed.
  \item Chapter~\ref{chap:archimpl} provides an in-depth view of the 
high-level architectural implementation.
  \item Chapter~\ref{chap:conclusions} wraps up the overall project experience.
 \end{itemize}
