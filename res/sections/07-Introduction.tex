\chapter{Introduction}
\label{chap:intro}

Through the history of Internet, many factors changed and new innovations came,
but at the end of the day a fundamental key remains: connection performances
are, indeed, what the users really want. Today, connectivity plays an important
role in everyone's life. Several decades brought solutions with enhanced wired
and wireless connectivity, like Ethernet, Fast Ethernet, 2G, 3G, LTE, and better
communication protocols such as a whole plethora of TCP standards in order to
keep up with the continuous surging of network traffic. Internet Service
Providers to be able to efficiently process the whole quantity of packets
passing through the network, applies several optimization along the packet path.
Until this day, these optimization have mostly been based on costly, in hardware
solutions. In particular, these Network Functions are able to perform only
generic packet elaboration, since they do not know the Quality of Service
required by every connection. It is in this context that 5G proposes a new way
to approach networking, suggesting the virtualization of this Network Functions,
making their deployments faster and cheaper. On top of that, it proposes
specific services for determined kind of data. If in the past a train running a
300km/h had to use the same network services of an IoT device sending data
updates every hour, now 5G proposes two different and efficient solutions for
this completely different services. This separation is performed through the
creation of determined network components called Network Slices. This new
possibility of offering the network with specialized class of services, combined
with emerging virtualization technologies like cloud computing,
hypervisor-virtualized machine or container-based solutions enables the
deployment of Virtualized Network Functions within seconds running on generic
purpose or custom hardware. While there may be some performances degreadation,
it has been demonstrated that their time-to-deploy and the possibility of
automation they offers are unprecedented~\cite{nguyen2017sdn}. On top of that,
specialized software acts as a MANager and Orchestrator (MANO) for this VNFs,
regulating their lifecycle. While most of the current proposals are based on
hypervisor on cloud solutions, in this thesis we are going to see a
proof-of-concept of a part of a Network Slice, equipped with a custom MANO
solution, based on the new container-based technology, called Docker, on a
recent orchestrator framework: Kubernetes.
 
\section*{Document organization}
 
This thesis is organized as follows:
\begin{itemize}
 \item Chapter~\ref{chap:background} describes background information about 5G 
and the technologies used to build an ETSI compliant architecture
 \item Chapter~\ref{chap:rel_wk} gives an insight of correlated works on this 
field
 \item Chapter~\ref{chap:prjan} explains the analysis performed on the 
requirements provided and the design choices performed.
 \item Chapter~\ref{chap:archimpl} provides an in-depth view of the 
high-level architectural implementation.
 \item Chapter~\ref{chap:conclusions} wraps up the overall project experience.
\end{itemize}
