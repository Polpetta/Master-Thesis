\chapter{Introduction}
\label{chap:intro}

Through its history, the Internet has undergone many, rather slow, innovation
cycles, each bringing into the light its unique and important role in human
societal evolution. The Internet has effectively become a utility, embedded into
our everyday fabric which is more than often taken for granted. Connectivity
plays an important role and service continuity and performance is what the users
really strive for.

Several decades of research and engineering effort have broadened the
possibilities with which users access the network i.e., enhanced wired and
wireless connectivity, like Ethernet, Fast Ethernet, 2G, 3G, LTE, while enjoying
their preferred services. This pace of innovation is matched by a similar
evolution of the TCP/IP protocol standards, trying to keep up with the
continuous surge of network traffic. On their side, Internet Service Providers
(ISPs) rely on ad-hoc optimizations of the end-to-end service delivery path. To
this date, optimization is made in terms of physically placing middlebox
devices, carrying out specialized network functions, on the delivery path. Deep
packet inspection, firewalling, ad-hoc transport optimization are some examples
of such functionalities deployed either at aggregation or core-network points.

This modus operandi, however, is neither scalable nor cost-effective.
Maintenance operational costs, careful planning and dedicated costly hardware
are some of the side-effects. Yet, another more subtle symptom is that of
coupling software lifecycle to the hardware (middlebox) hindering innovation.
Thanks to recent developments, new possibilities have arisen. In particular, the
trend in virtualization and softwarization techniques have open the door to the
so-called network softwarization, allowing for a potentially faster pace of
innovation in the network-side by exploiting the software as a driving force.

The fifth-generation public mobile network architecture, commonly referred to as
5G, has been designed to support highly reliable high-performance communications
to satisfy the different requirements of all kinds of applications and services.
To achieve such flexibility, 5G assigns a central role to “network
softwarization”, i.e. the ability of the network to operate according to rules
and policies that may be controlled by software entities. By leveraging on
Software Defined Networking (SDN) and Network Function Virtualization (NFV), the
5G architecture heavily relies on software entities whose behaviour may be
easily customized to adapt to the requirements of different applications and
tenants. The 5G vision is centered around the idea of building tenant- or
service-specific networks on top of a collection of heterogeneous shared
physical infrastructures. This concept is referred to as network slicing.
Slicing is a key enabler for making the network able to support applications
with a wide range of different requirements, including:
\begin{enumerate*}[label=(\roman*)]
\item enhanced mobile broadband (eMBB);
\item ultra-reliable and low-latency communications (URLLC) and
\item massive machine-type communications (mMTC).
\end{enumerate*}

This new possibility of provisioning a network with specialized, heterogeneous
classes of services, combined with emerging virtualization technologies like
cloud computing, hypervisor-virtualized machine or container-based solutions,
enables the deployment of Virtualized Network Functions (VNFs) within seconds
running on generic purpose or custom hardware. While there may be some
performances degradation, it has been demonstrated that their time-to-deploy and
the possibility of automation they offer are unprecedented~\cite{nguyen2017sdn}.
On top of that, a specialized software entity acts as a MANager and Orchestrator
(MANO) for these VNFs, managing their lifecycle. While most of the current
proposals are based on a hypervisor solution, in this thesis we set the
objective of providing a proof-of-concept implementation of an Infrastructure
Management and Orchestration plane based on the new container-based technology,
called Docker, on a recent orchestrator framework: Kubernetes. Moreover,
building on recent published results of the 3GPP Release 15, we outline a
software architecture capable of end-to-end service orchestration exploiting
container-technology.
 
\section*{Document organization}
 
This thesis is organized as follows:
\begin{itemize}
 \item Chapter~\ref{chap:background} describes background information about 5G
   and the technologies used to build an ETSI compliant architecture;
 \item Chapter~\ref{chap:rel_wk} gives an insight of correlated works on this
   field;
 \item Chapter~\ref{chap:prjan} explains the analysis performed on the
   requirements provided and the overall project roadmap;
 \item Chapter~\ref{chap:archimpl} provides an in-depth view of the high-level
   architectural implementation;
 \item Chapter~\ref{chap:vnf_ns_impl} gives a brief explanation of the SFC
   implementation;
 \item Chapter~\ref{chap:test} describes the tests done on the platform;
 \item Chapter~\ref{chap:fw} illustrates possible future works;
 \item Chapter~\ref{chap:conclusions} wraps up the overall project experience.
\end{itemize}
