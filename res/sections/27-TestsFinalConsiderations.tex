\section{Final considerations}

In these tests is possible to see one critical aspect concerning an architecture
that follows the containerization approach: great flexibility in terms of
deployments. This possibility to deploy services in the form of SFCs and VNFs in
a very small amount of time is one of the key points required in the new 5G
architecture, that the tests above demonstrated this project is able to adhere.
Finally, the ability to rapidly redeploy software means smooth and easier
software development lifecycle, where developers have the possibility to
automatically push ``fresh'' software to production, letting the MANO and thus
the container orchestrator handling all the difficult and delicate tasks to put
software in a production environment, eliminating the possibility of human
errors and assuring longer up-times.