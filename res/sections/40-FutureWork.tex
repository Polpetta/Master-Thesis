\chapter{Future Work}
\label{chap:fw}

The realization of this architecture proposal is not finished yet. Indeed, a lot
of functionalities could be implemented and new components could be added. We
are now going to see possible future additions to improve the system.

\section{Security concerns}

One of the aspects that is usually not really given the required attention is
the one regarding security. In this context, the security we talk is the one
about the whole system, and not the single VNFs implementations. Fortunately,
Kubernetes is equipped with out-of-the-box access management options. A good way
to manage user authentication in the system should be integrate the Role Based
Access model provided by Kubernetes directly in Harbor. In fact, Harbor does not
come with security features, and the exposed APIs are usable by everyone. This
could lead to possible Denial of Service (DoS) attack of the disruption of NS
services by a malicious attacker.

Another security aspect which is important to provide is the isolation between
network slices. This isolation, in our implementation, should be achieved
clearly identifying sets of nodes (the different slices) and restricting their
possibility to talk and interact with the other sets. This kind of restriction
should be achieved using different policy groups through Kubernetes.

\section{Kubernetes federation}

A single Kubernetes cluster presents limitations in terms of number of nodes
available on a single cluster and regarding the geographical position of this
nodes. Indeed, nodes in a cluster should be located or in the same network or
geographically closer to each other, in order to ensure low latencies between
data exchange and Pods communications. An available solution, introduced in
version 1.9 of Kubernetes, is the possibility to create federated Kubernetes
clusters. This solution ensures a proper division between groups of nodes, and
the possibility to ensure better performances creating clusters only with nodes
closer to each other. Unfortunately, we were not able to create a proper
Kubernetes federation. One of the main reasons was related the current
immaturity of the project, that could have created and unstable cluster
federation.

\section{Dynamic chain update}

The importance of having NS services updated as smooth as possible is without
doubts a feature that would ensure a continue service availability avoiding its
disruption. This kind of approach, though, requires particular attention about
keeping the current connections alive. This means at least keeping the old chain
up while deploying the new one, and gracefully redirect all the new traffic from
the chain to replace, until none is left. Only after this operation, the old
chain definition can be completely decommissioned and the related VNFs removed.
In our current implementation, the NS update functionality relies on a simpler
approach: when an update gets ordered, Kubernetes gets instructed to remove the
old chain and deploy the new one. Meanwhile, all the Roulette controllers get
updated with the new NS route. While this keeps the system implementation
simpler the back of the medal of this solution is that packets currently
travelling along the old NS could be lost during the chain update.
