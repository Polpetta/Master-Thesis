%**************************************************************
% Abstract
%**************************************************************
\cleardoublepage
\phantomsection

\thispagestyle{empty}

\newenvironment{changemargin}[2]{%
\begin{list}{}{%
\setlength{\topsep}{0pt}%
\setlength{\leftmargin}{#1}%
\setlength{\rightmargin}{#2}%
\setlength{\listparindent}{\parindent}%
\setlength{\itemindent}{\parindent}%
\setlength{\parsep}{\parskip}%
}%
\item[]}{\end{list}}

\vspace*{\fill}
\begin{changemargin}{1.5cm}{1.5cm}
\section*{Abstract}
The importance of Internet is growing every year. Mobile and wired traffic are
continuously surging, asking for better performances. On this day, ISPs rely on
optimized, ad-hoc and often in-hardware nework solutions. These network
functions are costly and require time to be deployed and mantained. In this
context, upcoming virtualization technolgies like hypervisor-based and
container-based are providing the possibility to create virtualized network
functions, that have the ability to be up and running in a matter of seconds.
VNFs are the managed and orchestrated by a specific application, the MANO that
handles their lifecycle.

In this thesis we are going to propose an Infrastructure Management and
Orchestrator plane Proof-of-Concept, based on the container virtualization
technology and on the 3GPP Release 15 specifications. In the current PoC, Docker
and the container orchestrator Kubernetes play a fundamntal role.
\vspace*{\fill}
\end{changemargin}
%\bigskip
