%**************************************************************
% Abstract
%**************************************************************
\cleardoublepage
\phantomsection

\thispagestyle{empty}

\vspace*{\fill}
\par
\begingroup
\leftskip4em
\rightskip\leftskip
\section*{\centering Abstract}
Network resource management and service optimization are two basic
functionalities widely adopted by network operators. However, provisioning these
two key capability is becoming increasingly difficult considering the high
number of services delivered through today networks. The usual approach of
optimization through middleboxes deployed in the end-to-end path between
user/service is proving incapable in keeping up with the current pace of change.
On the other side, advances in virtualization and softwarization techniques
could provide the much needed flexibility, serving as enablers for an
end-to-end, softwarized network management plane. In this context, taking
inspiration from the recent proposal of the 3GPP Release 15, we propose a
two-layer service orchestration architecture exploiting container technology. In
specifics, we provide a proof of concept implementation of an infrastructure
Management and Orchestration (MANO) plane, replying on the Kubernetes container
orchestration framework.
\par
\endgroup
\vspace*{\fill}
%\bigskip
