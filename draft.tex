\documentclass[10pt]{book}

\usepackage{todonotes}

\author{Polonio Davide}
\title{A thesis draft}
% The outline'd be something like this:
%  Thesis outline
% 
% - Introduction about the VIBES project (context, 5G, the need for a new 
% network architecture, proposed solution, requisites)
%   + Related works
% - Analysis of the available technologies (kubernetes, openstack, openvswitch, 
% Docker, Docker swarm)
%   + Choseen tecnhologies and description of the 'reason why'
% - Project description
%   + Initial solution with Openstack and Tacker
%   + From Tacker to Openbaton
%   + From Openbaton with Openstack to Openbaton with Docker
%   + Exploration of Docker orchestrators (Swam vs Kubernetes)
%   + Kubernetes internal composition (GlusterFS, Ingress, Harbor, metrics)
%   + Openbaton and Kubernetes integration (lack of a implemented Driver, 
% integration with existinf VNFM)
% - Case study
%   + Analysis of TCP/UDP performance
% - Results
% - Future works
% - Conclusions

\makeindex
\begin{document}
 \maketitle
 

 % This is just a draft, nothing suitable for the final product: sources are in 
 % form of a footnote
 
 % --- INTRODUCTION ---
\chapter{Introduction}

 Through the history of internet, many factor changed and new innovations came,
but at the end of the day a foundamental key remained: TCP/IP performance are, 
indeed, what the users unconsciously want.\todo{Idea: talk about how TCP/IP 
performes quite well in wired connections, while in wireless one, such as 
wireless/satellite, sucks ass} In this context network optimization plays an 
important role, and it is for this reason that network packets are processed 
while traversing the network in order to make their transmission as fast as 
possible. On top of that, mobile traffic is continuously surging every year, 
with the backbones having more and more data to process. In this context, 
in-hardware VNF solutions are becoming less efficent. Their time-to-deploy, in 
fact, is very high compared to the requirements of the network. With this 
trend, in-hardware solutions will not be able to keep up. Is with this 
considerations that software solutions are being proposed to become the new way 
to deploy network components. While less efficent, they are faster to bring them 
up, offering an unprecedented flexibility\footnote{SDN/NFV-Based Mobile Packet 
Core Network Architectures: A Survey}. Furthermore, new virtualization 
technologies such as Cloud computing and Docker are becoming more and more 
mature for being employed in large-scale production environments, where new 
frameworks, called orchestrators (e.g. Kubernetes, Docker Swarm), allow to 
easily manage the application lifecyle, i.e. deployment, scale up/down
and finally, removal.

\section{The VIBES project}
 
 This thesis is part of the VIBES project\todo{Talk about VIBES project, add
some reference, explain what it is.}, where the necessity for better TCP/IP
transmission through satellite connections is the main requirement. To
reach this goal, the project specifications suggest to exploit the 5G incoming
technology and use the NFV-MANO architecture to perform first packets
elaboration and performance improvment and finally TCP/IP satellite chunk
optimization with the Performance Enhancing Proxy (PEP). The VIBES project
proposed five technical requirements:
\begin{enumerate}
 \item Analysis of the applicability of current and new Internet protocols in
the proposed VNF-PEP architecture
 \item Implementation of a VNF-PEP prototype
 \item Building of a PoC test platform
 \item VNF-PEP validation and performance tests on 5G use-cases
 \item Demonstration testbed management
\end{enumerate}

\subsection{VNF-PEP architecture and internet protocols}

The analysis of this topic revealed to be trivial: since internet has many
different protocols that would become infeasible to support all of them at the
same time, packet incapsulation present itself as the only faseable solution:
every packet incoming in the VNFs has already been incapsulated by the MANO,
\todo{Not sure about this, please give it a check} making the whole
architecture indipendent from the protocol a particolar flux of data uses. To
achieve this, TUN/TAP\todo{Find out acronym} interfaces were used to deploy a
tunnel between VNFs, allowing them to pass packets encapsulated in UDP ones.

\subsection{VNF-PEP prototype}

Since our goal shifted into creating a MANO testbed, our prototype doesn't
include PEP. The VNF architecture was shaped following the container
orchestrator we decided to use: Kubernetes.

\subsubsection{About Kubernetes}

As already introduced before, Kubernetes is an open source software framework
specialized in container management and orchestration. Developed by Google in
??\todo{Find out the actual date} and written in Go, it allows different
machines (called nodes from now on) to create an abstraction layer and to form
a cluster where is possible to run Docker containers without brothering about
hardware constraints. Nodes can have specific roles and specialization, all of
them managed by one or many master nodes, that actually manage the
orchestration.
On top of that, Kubernetes is virtualization-agnostic, meaning that it offers
the possibility to change the virtualization software in use (i.e. from
Container-based technology to Virtual machine software).\todo{Add Kubernetes
logo}

\subsubsection{About Docker}

In recent years, with the new hardware capabilities and the recent development
of in-kernel virtualization systems (such as Hyper-V) this technology
begin to be adopted. Virtualization allows to run different systems on the
same machine, making them completely isolated and then more resilient to
failures. The back of the medal, though, is that virtual machines require large
amount of resources, especially memory, because solutions like copy-and-write
of in-memory pages are not viable anymore (since the kernel gets duplicated
too), leading to duplication of loaded libraries and assets in the memory of
the host.
For large deployment containing only simple services (e.g. a backend
application serving a website or a database) this ends up in a waste of
resources. \todo{Talk about lxc containers and how Docker solved magically the
problem.} It's here that, in ??\todo{Find Docker date of birth}, Docker was
created, basing its solution on an already existing product: Linux Container
(LXC). From this framework, Docker built an entire ecosystem, consisting in a
client/server model giving the possibility for users to simply launch, scale and
delete containers (locally or remotely with Docker-machine), a repository
system where images, a layer-based ``core'' where containers start running from
when they are launched, can be stored and tagged in a sort of version control
system. Finally, another couple of solutions, called Docker-compose, and Docker
Swarm have the purpose to, respectively, orchestrate containers in a single or
a clustered system. \todo{Add Docker, docker-machine logos}

\paragraph{Docker Swarm} Introduced in ??\todo{Find Docker Swarm date of
birth}, Docker Swarm allows multiple Docker nodes to cluster together and be
seen as one logical unit. This layer completely makes the underlying
infrastructure transparent: data management, as network one, are totaly managed
by Docker. Recently, Docker-compose support has been introduced, making
this one-node tool available for clustered systems. At the end of the day, on
one hand Docker Swarm provides an easy way to set up a cluster system with all
the tools configured out-of-the-box, without the necessity to set networking
configurations or installing ad-hoc storage solutions. On the other hand,
though, it doesn't have all the personalization options that Kuberentes offers,
thus making the product less flexible. This foundamental characteristic,
although, makes the two solutions have different use cases leading to different
market needs. \todo{Add Docker Swarm}

\paragraph{Docker-compose} As briefly discussed above, Docker-compose is a tool
that allows services orchestration. It automates most of the tasks that should
be performed by hand when launching one or multiple docker containers. It's
particulary useful when more continers have to interact together,
because\todo{Write down docker-commpose functionalities and describe the product
a little bit} all the services rules are described in a simple YAML file.
With the birth of Docker Swarm, work has started to port this tool to the Swarm
framework. At the time of this writing, the procedure to port a Docker-compose
configuration in Docker Swarm is still not completely transparent, and it
requires a sort of compilation, where Docker-compose bundles all the
necessaries resources into a binary file that Docker Swarm will use to allocate
the necessary resources.

\vspace{0.5cm}

Starting with the first step of creating a MANO able to process incoming data
packets through VNF functions, we encountered that many networking tools
already present in the market required some tweaking and some integrations,
shifting our goal to create a complete  European Telecommunications Standards
Institutes (ETSI) Management and orchestrator (MANO) testbed instead, following
the specifications suggested in the RFC 7665, thus implementing only the first
three requisites, without digging in the satellite data flow optimization.
\todo{This section needs to be expanded a lot!!!}


\subsubsection{Other technologies}\todo{Talk somehow of the other tecnhologies}
\end{document}
