\documentclass[10pt]{book}

\author{Polonio Davide}
\title{A thesis draft}
% The outline'd be something like this:
%  Thesis outline
% 
% - Introduction about the VIBES project (context, 5G, the need for a new 
% network 
% architecture, proposed solution, requisites)
%   + Related works
% - Analysis of the available technologies (kubernetes, openstack, openvswitch, 
% docker, docker swarm)
%   + Choseen tecnhologies and description of the 'reason why'
% - Project description
%   + Initial solution with Openstack and Tacker
%   + From Tacker to Openbaton
%   + From Openbaton with Openstack to Openbaton with Docker
%   + Exploration of docker orchestrators (Swam vs Kubernetes)
%   + Kubernetes internal composition (GlusterFS, Ingress, Harbor, metrics)
%   + Openbaton and Kubernetes integration (lack of a implemented Driver, 
% integration with existinf VNFM)
% - Case study
%   + Analysis of TCP/UDP performance
% - Results
% - Future works
% - Conclusions

\makeindex
\begin{document}
 \maketitle
 

 % This is just a draft, nothing suitable for the final product
 
 % --- INTRODUCTION ---
 \chapter{Introduction}
 
 In the modern mobile networking, these is the continuous necessity to 
improve the overall speed connection: important goals are low latency and a 
good throughtput. Mobile data is continuously sourging every year, with the 
backbones having more traffic to process. In this context, in-hardware backbone 
solutions are becoming less efficent. Their time-to-deploy, in fact, is very 
high compared to the requirement of the network. With this trend, in-hardware 
solutions will not be able to keep the 
 
 \section{The VIBES project}
\end{document}
