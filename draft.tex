\documentclass[10pt]{book}

\usepackage{todonotes}

\author{Polonio Davide}
\title{A thesis draft}
% The outline'd be something like this:
%  Thesis outline
% 
% - Introduction about the VIBES project (context, 5G, the need for a new 
% network 
% architecture, proposed solution, requisites)
%   + Related works
% - Analysis of the available technologies (kubernetes, openstack, openvswitch, 
% docker, docker swarm)
%   + Choseen tecnhologies and description of the 'reason why'
% - Project description
%   + Initial solution with Openstack and Tacker
%   + From Tacker to Openbaton
%   + From Openbaton with Openstack to Openbaton with Docker
%   + Exploration of docker orchestrators (Swam vs Kubernetes)
%   + Kubernetes internal composition (GlusterFS, Ingress, Harbor, metrics)
%   + Openbaton and Kubernetes integration (lack of a implemented Driver, 
% integration with existinf VNFM)
% - Case study
%   + Analysis of TCP/UDP performance
% - Results
% - Future works
% - Conclusions

\makeindex
\begin{document}
 \maketitle
 

 % This is just a draft, nothing suitable for the final product
 
 % --- INTRODUCTION ---
 \chapter{Introduction}

  Through the history of internet, many factor changed and new innovations came,
but at the end of the day a foundamental key remained: TCP/IP performance are, 
indeed, what the users unconsciously want.\todo{Idea: talk about how TCP/IP 
performes quite well in wired connections, while in wireless one, such as 
wireless/satellite, sucks ass} In this context network optimization plays an 
important role, and it is for this reason that network packets are processed 
while traversing the network in order to make their transmission as fast as 
possible. On top of that, mobile traffic is continuously surging every year, 
with the backbones having more and more data to process. In this context, 
in-hardware VNF solutions are becoming less efficent. Their time-to-deploy, in 
fact, is very high compared to the requirements of the network. With this trend, 
in-hardware solutions will not be able to keep up. Is with this considerations 
that software solutions are being proposed to become the new way to deploy 
network components. While less efficent, they are faster to bring them up, 
offering an unprecedented flexibility. Furthermore, new virtualization 
technologies such as Docker are becoming more and more mature for being employed 
in large-scale production environments, where new frameworks, called 
orchestrators (e.g. Kubernetes, Docker Swarm), allow to easily manage the 
application lifecyle, i.e. deployment, scale up/down
and finally, removal.

 \section{The VIBES project}
 
 This thesis initially begun with the goals descripted in the VIBES\todo{Talk 
about VIBES project, add some reference, explain what it is.} project, where 
the necessity for better TCP/IP transmission through satellite connections was 
the main requirement. Starting from that, we encountered many that, many 
networking tools already present in the market, required some tweaking and some 
integrations, shifting our goal to create a complete testbed for software VNF 
communications, following the specifications suggested in the RFC 7665.
\end{document}
