% Interlinea
\renewcommand{\baselinestretch}{1.1}

%**************************************************************
% file contenente le impostazioni della tesi
%**************************************************************

%**************************************************************
% Frontespizio
%**************************************************************
\newcommand{\myName}{Davide Polonio}                                    % autore
\newcommand{\myTitle}{5G as a Service: an ETSI-NFV compliant architecture proposal}                                 % tipo di tesi
\newcommand{\mySupervisor}{Armir Bujari}                        % relatore
\newcommand{\myCoSupervisor}{Name Surname}                      % co-relatore
% decomment line 12 or 13
\newbool{ifCoSupervisor}
\boolfalse{ifCoSupervisor}
% \boolfalse{ifCoSupervisor}
\newcommand{\myDegree}{Tesi di laurea magistrale}                
\newcommand{\myUni}{Università degli Studi di Padova}           % università
\newcommand{\myFaculty}{Corso di Laurea Magistrale in Informatica} % facoltà
\newcommand{\myDepartment}{Dipartimento di Matematica \\ ``Tullio Levi-Civita''}
\newcommand{\myLocation}{Padova}                                % dove
\newcommand{\myAA}{2017-2018}                                   % anno
\newcommand{\myTime}{December 2018}                             % quando


%**************************************************************
% Impostazioni di impaginazione
% see: http://wwwcdf.pd.infn.it/AppuntiLinux/a2547.htm
%**************************************************************

\setlength{\parindent}{14pt}   % larghezza rientro della prima riga
\setlength{\parskip}{0pt}   % distanza tra i paragrafi


%**************************************************************
% Impostazioni di biblatex
%**************************************************************
\bibliography{res/bibliography} % database di biblatex

\defbibheading{bibliography}
{
    \cleardoublepage
    \phantomsection
    \addcontentsline{toc}{chapter}{\bibname}
    \chapter*{\bibname\markboth{\bibname}{\bibname}}
}

\setlength\bibitemsep{1.5\itemsep} % spazio tra entry

% \DeclareBibliographyCategory{opere}
% \DeclareBibliographyCategory{web}
% 
% \addtocategory{web}{site:agile-manifesto}
% \addtocategory{web}{site:aws-lambda}
% \addtocategory{web}{site:aws-api-gateway}
% \addtocategory{web}{site:apex}
% \addtocategory{web}{site:swagger}
% \addtocategory{web}{site:messenger-facebook}
% \addtocategory{web}{site:aws-dynamodb}
% \addtocategory{web}{site:mocha}
% 
% \defbibheading{web}{\section*{Siti Web consultati}}


%**************************************************************
% Impostazioni di caption
%**************************************************************
\captionsetup{
    tableposition=top,
    figureposition=bottom,
    font=small,
    format=hang,
    labelfont=bf
}

%**************************************************************
% Impostazioni di glossaries
%**************************************************************
\newglossaryentry{nosql} {
  name=NoSQL,
  description={
NoSQL (Not Only SQL) è il nome di un tipo di DBMS che non prevede soltanto
l'utilizzo del modello relazionale utilizzato dai sistemi classici di tipo SQL.
}
}

\newglossaryentry{javascript_object_notation} {
  name=JavaScript Object Notation,
  description={
Formato adatto all’interscambio di dati fra applicazioni client-server. La
semplicità di JSON ne ha decretato un rapido utilizzo specialmente nella
programmazione in AJAX (Asynchronous JavaScript and XML). Il suo uso tramite
JavaScript è particolarmente semplice e questo fatto lo ha reso velocemente
molto popolare.
}
}

\newglossaryentry{url} {
  name=URL,
  description={
URL (Uniform Resource Locator) indica, specificando la posizione di una macchina
collegata alla rete internet e il meccanismo per ottenerla, un indirizzo a una
risorsa web.
}
}

\newglossaryentry{crud} {
  name=CRUD,
  description={
CRUD (Create, Read, Update and Delete) indica le operazioni effettuabili su
dispositivi di memoriazzazione. Nel contesto web, questo termine è stato
associato alle operazioni del protocollo HTTP.
}
}

\newglossaryentry{cloud} {
  name=Cloud,
  description={
Il cloud computing è un tipo di computing basato su internet, che provvede a
fornire risorse e informazioni ad altri computer su richiesta. È un modello di
computing che permette di configurare risorse in maniera rapida, basandosi
sulla condivisione di potenza di calcolo e virtualizzazione. Questo tipo di
modello permette ad aziende di creare la propria infrastruttura in maniera
flessibile, senza avere degli alti costi iniziali e gestionali.
}
}

\newglossaryentry{serverless} {
  name=Architettura serverless,
  description={
Un'architettura serverless è un tipo di cloud computing dove il gestore del
servizio cloud gestisce l'avvio e lo spegnimento delle macchine virtuali
necessarie per gestire le richieste, le risorse necessarie per il loro
funzionamento e la manutenzione dei sistemi. Questo, a differenza del nome, non
indica il fatto che non siano presenti server, ma indica il fatto che colui che
usufruisce di questo tipo di servizio non deve comprare o affittare alcun tipo
di macchine (reali o virtuali) per eseguire il proprio codice.
}
}

\newglossaryentry{api} {
  name=API,
  description={
Acronimo di \textit{Application Programming Interface}, indica un insieme
di procedure, funzionalità e protocolli per costruire programmi e applicazioni.
}
}

\newglossaryentry{yaml} {
  name=YAML,
  description={
Acronimo di \textit{YAML Ain't Markup Language}, ed è un formato file
per la serializzazione di dati utilizzabile dagli esseri umani. Nato nel 2001,
è stato progettato per l'inserimento di dati sottoforma di liste e array
associativi. È soprattutto impiegato nei file di configurazione e
presenta una buona facilità di lettura.
}
}

\newglossaryentry{agile} {
  name=Agile,
  description={
Insieme di principi per lo sviluppo software sotto i quali requisiti e
soluzioni evolvono tramite gli sforzi collettivi del gruppo auto-organizzato
polifunzionale.
}
}

\newglossaryentry{scrum} {
  name=Scrum,
  description={
SDLC iterativo-incrementale concepito per rilasci brevi e programmati del
software. L'obiettivo di Scrum è essere flessibile al cambio dei requisiti
imposti dal cliente, in modo da poter soddisfare al meglio le richieste
mantenendo un controllo sulla qualità e le tempistiche del progetto.
}
}

%Definizione presa da:
%https://it.wikipedia.org/wiki/Scrum_(informatica)
\newglossaryentry{scrumMaster} {
  name=Scrum Master,
  description={
Colui che facilita una corretta esecuzione dei processi di sviluppo. Lo Scrum
Master detiene l'autorità relativa all'applicazione delle norme, spesso
presiede le riunioni importanti e pone sfide alla squadra per migliorarla.
}
}

\newglossaryentry{startup}
{
  name=Startup,
  text=startup,
  sort=startup,
  description={
In economia, con questo termine, si indica una nuova impresa
nelle forme di un'organizzazione temporanea o una società di capitali in cerca
di un business model ripetibile e scalabile.
La scalabilità è un elemento cardine di questa tipologia di impresa. L'avvio di
un'attività imprenditoriale non scalabile, come l'apertura di un ristorante,
non coincide dunque con la creazione di una startup ma, piuttosto, di una
società tradizionale
}
}


% Acronimi
\newacronym{ceo}{CEO}{Chief Executive Officer}
\newacronym{cto}{CTO}{Chief Technology Officer}
\newacronym{ram}{RAM}{Random Access Memory}
\newacronym{rest}{REST}{Representational State Transfer}
\newacronym{soap}{SOAP}{Simple Object Access Protocol}
\newacronym{xml}{XML}{Extensible Markup Language}
\newacronym{http}{HTTP}{Hypertext Transfer Protocol}
\newacronym{mime}{MIME}{Multipurpose Internet Mail Extensions}
\newacronym{arn}{ARN}{Amazon Resource Name}
\newacronym{npm}{NPM}{Node Package Manager}
 % database di termini
\makeglossaries


%**************************************************************
% Impostazioni di graphicx
%**************************************************************
\graphicspath{{res/img/}} % cartella dove sono riposte le immagini


%**************************************************************
% Impostazioni di hyperref
%**************************************************************
\hypersetup{
    %hyperfootnotes=false,
    %pdfpagelabels,
    %draft,	% = elimina tutti i link (utile per stampe in bianco e nero)
    colorlinks=true,
    linktocpage=true,
    pdfstartpage=1,
    pdfstartview=FitV,
    % decommenta la riga seguente per avere link in nero (per esempio per la
%stampa in bianco e nero)
    %colorlinks=false, linktocpage=false, pdfborder={0 0 0}, pdfstartpage=1,
%pdfstartview=FitV,
    breaklinks=true,
    pdfpagemode=UseNone,
    pageanchor=true,
    pdfpagemode=UseOutlines,
    plainpages=false,
    bookmarksnumbered,
    bookmarksopen=true,
    bookmarksopenlevel=1,
    hypertexnames=true,
    pdfhighlight=/O,
    %nesting=true,
    %frenchlinks,
    urlcolor=webbrown,
    linkcolor=webbrown,
    citecolor=webgreen,
    %pagecolor=RoyalBlue,
    %urlcolor=Black, linkcolor=Black, citecolor=Black, %pagecolor=Black,
    pdftitle={\myTitle},
    pdfauthor={\textcopyright\ \myName, \myUni, \myFaculty},
    pdfsubject={},
    pdfkeywords={},
    pdfcreator={pdfLaTeX},
    pdfproducer={LaTeX}
}

%**************************************************************
% Impostazioni di itemize
%**************************************************************
% \renewcommand{\labelitemi}{$\ast$}

%\renewcommand{\labelitemi}{$\bullet$}
%\renewcommand{\labelitemii}{$\cdot$}
%\renewcommand{\labelitemiii}{$\diamond$}
%\renewcommand{\labelitemiv}{$\ast$}


%**************************************************************
% Impostazioni di listings
%**************************************************************
%\lstset{
%    language=[LaTeX]Tex,%C++,
%    keywordstyle=\color{RoyalBlue}, %\bfseries,
%    basicstyle=\small\ttfamily,
%    %identifierstyle=\color{NavyBlue},
%    commentstyle=\color{Green}\ttfamily,
%    stringstyle=\rmfamily,
%    extendedchars=true,
%    numbers=left,%
%    numberstyle=\tiny,
%    stepnumber=1,
%    numbersep=8pt,
%    showstringspaces=false,
%    breaklines=true,
%    frameround=ftff,
%    frame=single
%}
\definecolor{codegreen}{RGB}{0,153,0}
\definecolor{codegray}{RGB}{96,96,96}
\definecolor{codered}{RGB}{204,0,0}
\definecolor{codeblue}{RGB}{0,0,153}
\definecolor{codelightblue}{RGB}{0,204,204}
\definecolor{codepurple}{RGB}{0.58,0,0.82}
\definecolor{backcolour}{RGB}{235,235,235}

\newcommand\YAMLcolonstyle{\color{black}\footnotesize}
\newcommand\YAMLkeystyle{\color{blue}\footnotesize}
\newcommand\YAMLvaluestyle{\color{black}\footnotesize}

\makeatletter

% here is a macro expanding to the name of the language
% (handy if you decide to change it further down the road)
\newcommand\language@yaml{yaml}

\expandafter\expandafter\expandafter\lstdefinelanguage
\expandafter{\language@yaml}
{
  keywords={true,false,null,y,n},
  keywordstyle=\color{codeblue}\bfseries,
  basicstyle=\YAMLkeystyle\ttfamily,                                 % assuming a key comes first
  sensitive=false,
  comment=[l]{\#},
  morecomment=[s]{/*}{*/},
  commentstyle=\color{codegrey}\ttfamily,
  stringstyle=\YAMLvaluestyle\ttfamily,
  moredelim=[l][\color{codered}]{\&},
  moredelim=[l][\color{codered}]{*},
  moredelim=**[il][\YAMLcolonstyle{:}\YAMLvaluestyle]{:},   % switch to value style at :
  morestring=[b]',
  morestring=[b]",
  literate =    {---}{{\ProcessThreeDashes}}3
                {>}{{\textcolor{codered}\textgreater}}1     
                {|}{{\textcolor{codered}\textbar}}1 
                {\ -\ }{{\ -\ }}3,
}

% switch to key style at EOL
\lst@AddToHook{EveryLine}{\ifx\lst@language\language@yaml\YAMLkeystyle\fi}
\makeatother

\newcommand\ProcessThreeDashes{\hspace*{0.4cm}\llap{\color{codered}\footnotesize-{-}-}}

\colorlet{punct}{red!60!black}
\definecolor{background}{HTML}{EEEEEE}
\definecolor{delim}{RGB}{20,105,176}
\colorlet{numb}{magenta!60!black}

\lstdefinelanguage{json}{
    %basicstyle=\normalfont\ttfamily,
    %numbers=left,
    %numberstyle=\scriptsize,
    %stepnumber=1,
    %numbersep=8pt,
    showstringspaces=false,
    breaklines=true,
    %frame=lines,
    %backgroundcolor=\color{background},
    literate=
     *{0}{{{\color{numb}0}}}{1}
      {1}{{{\color{numb}1}}}{1}
      {2}{{{\color{numb}2}}}{1}
      {3}{{{\color{numb}3}}}{1}
      {4}{{{\color{numb}4}}}{1}
      {5}{{{\color{numb}5}}}{1}
      {6}{{{\color{numb}6}}}{1}
      {7}{{{\color{numb}7}}}{1}
      {8}{{{\color{numb}8}}}{1}
      {9}{{{\color{numb}9}}}{1}
      {:}{{{\color{punct}{:}}}}{1}
      {,}{{{\color{punct}{,}}}}{1}
      {\{}{{{\color{delim}{\{}}}}{1}
      {\}}{{{\color{delim}{\}}}}}{1}
      {[}{{{\color{delim}{[}}}}{1}
      {]}{{{\color{delim}{]}}}}{1},
}

\lstdefinestyle{mystyle}{
    basicstyle={\footnotesize\ttfamily},
    backgroundcolor=\color{backcolour},   
    commentstyle=\color{codegray},
    keywordstyle=\color{codeblue},
    numberstyle=\tiny\color{codegray},
    stringstyle=\color{codegreen},
    breakatwhitespace=false,         
    breaklines=true,                 
    captionpos=b,                    
    keepspaces=true,                 
    numbers=left,                    
    numbersep=5pt,                  
    showspaces=false,                
    showstringspaces=false,
    showtabs=false,                  
    tabsize=2
}
 
\lstset{style=mystyle}


%**************************************************************
% Impostazioni di xcolor
%**************************************************************
\definecolor{webgreen}{rgb}{0,.5,0}
\definecolor{webbrown}{rgb}{.6,0,0}
\definecolor{Pantone}{RGB}{155,0,20}
\definecolor{GrigioLight}{RGB}{152, 152, 152}

%**************************************************************
% Altro
%**************************************************************

\newcommand{\omissis}{[\dots\negthinspace]} % produce [...]

\newcommand{\sectionname}{section}
\addto\captionsitalian{\renewcommand{\figurename}{figure}
                       \renewcommand{\tablename}{table}}

\newcommand{\glsfirstoccur}{\ap{{[g]}}}

\newcommand{\intro}[1]{\emph{\textsf{#1}}}

%**************************************************************
% Environment per ``namespace description''
%**************************************************************

\newenvironment{namespacedesc}{
    \vspace{10pt}
    \par \noindent                              % start new paragraph
    \begin{description}
}{
    \end{description}
    \medskip
}

\newcommand{\classdesc}[2]{\item[\textbf{#1:}] #2}


%-------------------INIZIO creazione subsubparagraph---------------------------
\makeatletter
\newcounter{subsubparagraph}[subparagraph]
\def\toclevel@subsubparagraph{6}
\renewcommand\thesubsubparagraph{%
  \thesubparagraph.\@arabic\c@subsubparagraph}
\newcommand\subsubparagraph{%
  \@startsection{subsubparagraph}    % counter
    {6}                              % level
    {\parindent}                     % indent
    {3.25ex \@plus 1ex \@minus .2ex} % beforeskip
    {-1em}                           % afterskip
    {\normalfont\normalsize\bfseries}}
\newcommand\l@subsubparagraph{\@dottedtocline{6}{13.5em}{5em}}
\newcommand{\subsubparagraphmark}[1]{}
\setcounter{tocdepth}{6}
\setcounter{secnumdepth}{6} % aggiunge contatore ai paragrafi
\makeatother
%-------------------FINE creazione subsubparagraph-----------------------------

%-------------------Capitoli personalizzati------------------------------------

\titleformat{\chapter}[display]
  {\normalsize \huge  \color{black}}%
  {\flushright\normalsize \color{Pantone}%
   \MakeUppercase{\chaptertitlename}\hspace{1ex}%
   {\fontsize{60}{60}\selectfont\thechapter}}%
  {10 pt}%
  {\bfseries\huge#1}%

%----------------FINE Capitoli personalizzati----------------------------------

%----------------INIZIO Parte personalizzata-----------------------------------

\renewcommand\thepart{\Alph{part}}

\newcommand\partnumfont{% font specification for the number
  \fontsize{304}{104}\color{white}\selectfont%
}

\newcommand\partnamefont{% font specification for the name "PART"
  \color{white}\huge\bfseries%
}

\titleformat{\part}[display]
   {\normalfont\huge\filleft}
   { }
   {20pt}
   {\begin{tikzpicture}[remember picture,overlay]
  \fill[GrigioLight]
    (current page.north west) rectangle ([yshift=-13cm]current page.north
east);
    \node[
      fill=Pantone,
      text width=2\paperwidth,
      rounded corners=6cm,
      text depth=12cm,
      anchor=center,
      inner sep=0pt] at ([yshift=21cm]current page.south west) (parttop)
    {\thepart};%
    \node[
      anchor=center,
      inner sep=0pt,
      outer sep=0pt] at ([xshift=16cm, yshift=6cm]parttop.south) (partnum)
    {\partnumfont\thepart};%
    \node[
      anchor=north east,
      align=right,
      inner sep=0pt] at ([yshift=10cm] current page.center)
    {\parbox{.7\textwidth}{\raggedleft\partnamefont\MakeUppercase{#1}}};
    \end{tikzpicture}%
}

%--------------- FINE parte personalizzata ------------------------------------

%--------------- MULTI columns ------------------------------------------------

\newcounter{countitems}
\newcounter{nextitemizecount}
\newcommand{\setupcountitems}{%
  \stepcounter{nextitemizecount}%
  \setcounter{countitems}{0}%
  \preto\item{\stepcounter{countitems}}%
}
\makeatletter
\newcommand{\computecountitems}{%
  \edef\@currentlabel{\number\c@countitems}%
  \label{countitems@\number\numexpr\value{nextitemizecount}-1\relax}%
}
\newcommand{\nextitemizecount}{%
  \getrefnumber{countitems@\number\c@nextitemizecount}%
}
\newcommand{\previtemizecount}{%
  \getrefnumber{countitems@\number\numexpr\value{nextitemizecount}-1\relax}%
}
\makeatother    
\newenvironment{AutoMultiColItemize}{%
\ifnumcomp{\nextitemizecount}{>}{3}{\begin{multicols}{2}}{}%
\setupcountitems\begin{itemize}}%
{\end{itemize}%
\unskip\computecountitems\ifnumcomp{\previtemizecount}{>}{3}{\end{multicols}}{}}

\newenvironment{AutoMultiColEnumerate}{%
\ifnumcomp{\nextitemizecount}{>}{3}{\begin{multicols}{2}}{}%
\setupcountitems\begin{enumerate}}%
{\end{enumerate}%
\unskip\computecountitems\ifnumcomp{\previtemizecount}{>}{3}{\end{multicols}}{}}

%--------------- END MULTI columns ---------------------------------------------

